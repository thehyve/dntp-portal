\documentclass{report}

\usepackage{prelude}

\title{DNTP security requirements and testing plan}
\begin{document}

\chapter{Security requirements}

\section{Introduction}
Scope limited to backend and infrastructure.
The consequence of this is that possible vulnerabilities of the
Angular frontend application are not considered.
This means that an authorised user may use the application, which
may leak the information the user is allows to access, or perform
actions the user is allowed to perform. E.g., a cross-site scripting
(XSS) attack scenario where a requester inserts Javascript code into
a request, which is opened and run by the browser of a Palga user.

\section{Security guidelines}

References:
\begin{itemize}

\item OWASP security testing guide \cite{owasp2013:testing};
\item Top 10 \cite{owasp2013:top10};
\item Application Security Verification Standard (ASVS) \cite{owasp2014:asvs}.
\end{itemize}

From the OWASP ASVS we target at complience with Level 2. The detailed requirements
in the standard \cite{owasp2014:asvs} apply here as well:
\begin{description}

\item[V2: Authentication] Requirements:
\begin{itemize}
\item Non-public pages require authentication (enforced server-side); 
\item Strong passwords; 
\item No weaknesses in account update/creation functionality; 
\item Enable to securely update credentials; 
\item Logging of authentication decisions (for security investigations);
\item Secure password hashing and storage;
\item No leaking of credentials;
\item No harvesting of user accounts possible through login functionality;
\item No default passwords active;
\item Blocking attacks that try multiple passwords and multiple accounts (V2.20);
\item `Forgot password' with expiring unique link, not sending an actual password;
\item No `secret questions';
\item Possibility to disallow a number of previously chosen passwords (V2.25).
\end{itemize}

\item[V3: Session Management] Requirements:
\begin{itemize}
\item Proper session management (invalidation, timeout, new session id after login,
only accept framework generated sessions; session tokens long enough and securely generated,
limited cookie domain);
\item Logout links;
\item No leaking of the session id in URLs;
\item HttpOnly (?);
\item Secure properties for session cookies (!);
\item Disallow duplicate sessions for a user originating from different machines (?).
\end{itemize}

\item[V4: Access Control] Requirements:
\begin{itemize}
\item Allow users only to secured functions, pages, files that they are authorised to use;
\item Allow users only to directly access and manipulate objects (with id reference),
for which they have permission;
\item Access controls fail securely;
\item Directory browsing disabled;
\item Verify consistency of client-side and server-side access control;
all access control enforced server-side;
\item Access control rules and data not changeable by users (e.g., changing ownership of data);
\item Logging of access control decisions (!).
\item Strong random tokens against Cross-site Request Forgery (CSRF/XSRF);
\item Aggregate access control protection  (V4.17), e.g., scraping by generating 
a large sequence of requests (a possibly authenticated user looking for security holes);
\end{itemize}



\item[V5: Malicious Input Handling] Requirements:
\begin{itemize}
\item No buffer overflow vulnerabilities (!);
\item UTF8 specified for input;
\item Input validation on server-side. Failure rejects input;
\item Protection agains injection attacks (SQL, OS);
\item Escape untrusted client data;
\item Protect sensitive fields (e.g., role, passwords) from automatic binding (V5.17);
\item Protection against parameter pollution attacks (V5.18).
\end{itemize}

\item[V7: Cryptography at Rest]
\begin{itemize}
\item Cryptography performed server-side;
\item Cryptography functions fail securely;
\item Access to master secret protected;
\item Random numbers generated for security (session id, password hashing,
XSRF tokens) generated by secure cryptographic functions;
\item Policy for managing cryptographic keys (creating, distributing, revoking, expiry).
\end{itemize}

\item[V8: Error Handling and Logging] 
\begin{itemize}
\item No logging of sensitive data, session id, personal details, credentials, stack traces;
\item Logging implemented server-side, only on trusted devices;
\item Error handling logic should deny access by default (e.g., throw exception which is not caught);
\end{itemize}


\item[V9: Data Protection]
\begin{itemize}
\end{itemize}


\item[V10: Communications Security]
\begin{itemize}
\end{itemize}


\item[V11: HTTP Security]
\begin{itemize}
\end{itemize}


\item[V13: Malicious Controls]
\begin{itemize}
\end{itemize}

\item[V15: Business Logic]
\begin{itemize}
\end{itemize}


\item[V16: Files and Resources]
\begin{itemize}
\end{itemize}


\item[V17: Mobile]
\begin{itemize}
\end{itemize}


\end{description}


\section{Specific requirements}

Security of data more important than availability.

Encrypting data.

\chapter{Security measures}

\section{Implementation and frameworks}

Meassures taken:
\begin{itemize}
\item Spring security for authentication and authorization;
\item Spring does CSRF prevention;
\item \texttt{mod\_security} in Apache?
\item Access to functionality in Spring routing;
\item Access to data in controller logic and using security annotations.
\item Using JPA derived queries in the repository classes 
(SQL queries are derived from interface method names), preventing SQL injection.
\item Input sanitisation by Spring (and Angular?)
\end{itemize}

Infrastructure:
\begin{itemize}
\item Application run in user space;
\item Apache proxy in between, application not directly approachable;
\item Only ports 80 and 443 open from the outside (firewall rules);
\item On port 80, Apache should only forward to https (rewrite rule, 
status 301 (Moved permanently)).
\item Access log by Apache, logging all incoming requests (URIs, not data).
\end{itemize}


\section{Security testing}
\begin{itemize}
\item Check logging of authentication decisions (!) and access to data.
\item Security code review:
\begin{itemize}
\item Proper routes, method annotation on all REST controllers?
\item How are user roles assigned? Who can change them?
\item No custom queries or all nicely prepared using prepared statements
(JPA provides this);
\item Check if user input is not used for:
demermining filesystem paths, checking for access, determining user id or roles,
database key in a query.%
\footnote{A common error: checking access based on the \emph{id} in the url
(\texttt{@PathVariable}),
but fetching the database object based on the \emph{id} in the request body.}
\end{itemize}
\item Test injection:
\begin{itemize}
\item Test for XSS attacks: try to insert HTML and Javascript in input fields;
this should be handled automagically by Spring.
\end{itemize}
\item Penetration testing?
\item Regularly check \url{https://pivotal.io/security} for vulnerabilities in Spring.
\end{itemize}


\section{Deployment checks}
Disable dev/test behaviour:
\begin{itemize}
\item No default users (e.g., \texttt{palga} -- check startup log of application and user administration);
\item No action \texttt{/test/clear/} available;
\item Check account blocking period after $n$ failed login attempts.
\item Check for SSL certificate, make sure key chain is secure and valid.
\item Check that the database is not accessible (except from localhost) and passwords are properly set.
\item Check that database password in the application configuration is not exposed (e.g., in \texttt{puppet}).
\item Perhaps write a script that checks if the current deployment uses the newest versions of included frameworks?
\end{itemize}


\section{Known weaknesses}
\begin{itemize}
\item Information sent plain text by email:
password recovery link.
\end{itemize}

\bibliographystyle{plain}
%\bibliographystyle{alpha}
\bibliography{bibliography}

\end{document}
